\documentclass[12pt]{book}

\title{Analysis II}
\author{Fall Semester\\MATH201}
\date{}
\usepackage{indentfirst}
\usepackage{kotex}
\usepackage{amssymb, amsthm, amsfonts, graphics, epsfig, fancyhdr, bm, mathrsfs, thmtools, hyperref, amsmath, inputenc, mathrsfs, mathtools, arydshln}
\usepackage[shortlabels]{enumitem}
\setlength{\headheight}{28pt}
\pagestyle{fancy}
\fancyhf{}
\fancyhead[L]{Written by Sung Jae Hyuk}
\fancyhead[R]{Ch8. Some Special Functions}
\fancyfoot[C]{\thepage}

\setcounter{chapter}{7}
\setlength\parindent{12pt}
\theoremstyle{definition}
\newtheorem{theorem}{Theorem}
\newtheorem{lemma}[theorem]{Lemma}
\newtheorem{corollary}[theorem]{Corollary}
\newtheorem{proposition}[theorem]{Proposition}
\newtheorem{remark}[theorem]{Remarks}
\newtheorem{definition}[theorem]{Definition}
\newtheorem{exe}{Exercise}
\newtheorem{sect}{Sec}

\usepackage{chngcntr}

\def\upint{\mathchoice%
	{\mkern13mu\overline{\vphantom{\intop}\mkern7mu}\mkern-20mu}%
	{\mkern7mu\overline{\vphantom{\intop}\mkern7mu}\mkern-14mu}%
	{\mkern7mu\overline{\vphantom{\intop}\mkern7mu}\mkern-14mu}%
	{\mkern7mu\overline{\vphantom{\intop}\mkern7mu}\mkern-14mu}%
	\int}
\def\lowint{\mkern3mu\underline{\vphantom{\intop}\mkern7mu}\mkern-10mu\int}

\renewcommand\arraystretch{1.3}
\renewcommand{\theequation}{\arabic{equation}}
\renewcommand{\thesect}{2.\arabic{sect}}
\renewcommand{\thetheorem}{3.\arabic{theorem}}
\renewcommand{\qedsymbol}{}
\newcommand{\N}{\mathbb{N}}
\newcommand{\Z}{\mathbb{Z}}
\newcommand{\Q}{\mathbb{Q}}
\newcommand{\R}{\mathbb{R}}
\newcommand{\C}{\mathbb{C}}

\makeatletter
\newcommand{\chapterauthor}[1]{%
	{\parindent0pt\vspace*{-25pt}%
	\linespread{1.1}\large\scshape#1%
	\par\nobreak\vspace*{35pt}}
	\@afterheading%
}
\newcommand{\chapterother}[1]{%
	{\parindent0pt\vspace*{-25pt}%
	\linespread{1.1}#1%
	\par\nobreak\vspace*{35pt}}
	\@afterheading%
}
\makeatother

\begin{document}
	\chapter{Some Special Functions\\Selected Exercise}
	\chapterauthor{Sung Jae Hyuk}
	\chapterother{Junior in Korea university\\ Majoring in computer science \& mathematics\\Email: okaybody10@korea.ac.kr}
	\listoftheorems[title = List of Exercise]
	\newpage
	\begin{lemma}
		Let $P_n(x)$ be set of polynomials of degree $3n+1$. Suppose $f(x)=e^{-1/x^2}$ for $x\neq 0$, then
		\begin{equation*}
			\displaystyle\lim_{x\rightarrow 0} g\left(\dfrac{1}{x}\right)f(x)=0
		\end{equation*} where $g \in P_n(x)$.
	\end{lemma}
	\begin{proof}
		Let $x=\dfrac{1}{t}$. It is easy to show that \begin{equation*}
			\displaystyle\lim_{x\rightarrow 0}g\left(\dfrac{1}{x}\right)f(x)=0
		\end{equation*} if and only if
		\begin{equation*}
			\displaystyle\lim_{t\rightarrow\infty}g(t)f\left(\dfrac{1}{t}\right)=\lim_{t\rightarrow-\infty}g(t)f\left(\dfrac{1}{t}\right)=0
		\end{equation*} using $\varepsilon-\delta$ argument.\\
		So, we will show that $\displaystyle\lim_{t\rightarrow \infty}g(t)f\left(\dfrac{1}{t}\right)=L.$\\
		As $\displaystyle\lim_{t\rightarrow \infty}\left\{t^{3n+2}-g(t)\right\}=\infty$, there exists $C>0$ such that $\left\vert g(t)\right\rvert \leq C\,t^{3n+2}$ for all $t> 0$.
		\begin{align*}
			|g(t)|\leq C\,t^{3n+2} &\Leftrightarrow -C\,t^{3n+2}\leq g(t) \leq C\,t^{3n+2}\\
			&\Leftrightarrow -C\,t^{3n+2}e^{-t^2}\leq g(t)f\left(\dfrac{1}{t}\right) \leq C\,t^{3n+2}e^{-t^2}\\
			&\Rightarrow \displaystyle\lim_{t\rightarrow \infty}-C\,t^{3n+2}e^{-t^2}\leq \lim_{t\rightarrow \infty}g(t)f\left(\dfrac{1}{t}\right) \leq \lim_{t\rightarrow\infty}C\,t^{3n+2}e^{-t^2}
		\end{align*}
		By squeeze theorem, $\displaystyle\lim_{ t\rightarrow \infty} g(t)f\left(\dfrac{1}{t}\right)=0$.\\
		The case $t\rightarrow-\infty$ is analogous, and proof is completed.
	\end{proof}
	\newpage
	\begin{exe}
		Define
		\begin{equation*}
			f(x) = 
			\begin{cases}
				e^{-1/x^2}&(x\neq 0),\\
				0&(x=0).
			\end{cases}
		\end{equation*}
	Prove that $f$ has derivatives of all orders at $x=0$, and that $f^{(n)}(0)=0$ for $n=1$, $2$, $3$, $\ldots$.
	\end{exe}
	\begin{proof}
		Let $f(x)=e^{-1/x^2}$ for all $x\neq0$, and $f(0)=0$.\\
		Then, we are enough to show that
		\begin{equation*}
			f^{(n)}(x)=
				\begin{cases}
					f(x)Q_n(x)&\quad (x>0)\\
					0&\quad (x=0)
				\end{cases}
		\end{equation*}
		where $P_n(x)$ a polynomial function of degree $n$, and $Q_n(x)=P_{3n}\left(\dfrac{1}{x}\right)$ fulfills the recursive definition
		\begin{align*}
			Q_0(x)&=1 \\
			Q_n(x)&=\dfrac{2}{x^3}Q_{n-1}(x)+Q_{n-1}'(x)
		\end{align*}
		Let us use mathematical induction.\\
		It is easy to show when $n=1$.\\
		Assume $n=k$ is true.\\
		Then, $f^{(k+1)}(x)$ is well-defined for $x\neq 0$.\\
		More Specifically,
		\begin{align*}
			f^{(k+1)}(x)&=\left\{e^{-1/x^2}\right\}'Q_k(x) + e^{-1/x^2}\left\{Q_k(x)\right\}'\\
			&=\dfrac{2}{x^3}\,e^{-1/x^2}Q_k(x)+e^{-1/x^2}\,Q_{k}'(x)\\
			&=e^{-1/x^2}\left\{\dfrac{2}{x^3}Q_k(x)+Q_k'(x)\right\}\\
			&=e^{-1/x^2}Q_{k+1}(x)\\
			&=f(x)Q_{k+1}(x)
		\end{align*}
		So if we show $f^{(k+1)}(0)=0$, then we can say that $f^{(k+1)}$ is also differentiable for $x\in\R$.\\
		By definition,
		\begin{flalign*}
			&&f^{(k+1)}(0)&=\displaystyle\lim_{x\rightarrow0}\dfrac{f^{(k)}(x)-f^{(k)}(0)}{x-0}&\\
			&&&=\lim_{x\rightarrow0}\dfrac{f^{(k)}(x)}{x}&(\because~f^{(k)}(0)=0)\\
			&&&=\lim_{x\rightarrow0}f(x)\dfrac{Q_k(x)}{x}\pagebreak\\
			&&&=\lim_{x\rightarrow0} f(x)P_{3k+1}\left(\dfrac{1}{x}\right)&(\because~Q_k(x)=P_{3k}\left(\dfrac{1}{x}\right))\\
			&&&=0&(\because~\text{By Lemma } 1)
		\end{flalign*}
		Hence $f^{(k+1)}(0)=0$, $f^{(k+1)}$ is also differentiable for $x\in\R$.\\
		By the principle of mathematical induction, $f(x)$ is infinitely differentiable for $x\in\R$.
	\end{proof}
	\newpage
	\begin{exe}
		Let $a_{ij}$ be the number in the $i$th row and $j$th column of the array
		\begin{equation*}
			\begin{matrix*}[r]
				-1&0&0&0&\cdots\\
				\frac{1}{2}&-1&0&0&\cdots\\
				\frac{1}{4}&\frac{1}{2}&-1&0&\cdots\\
				\frac{1}{8}&\frac{1}{4}&\frac{1}{2}&-1&\cdots\\
				\hdashline[2pt/2pt]
			\end{matrix*}
		\end{equation*}
		so that
		\begin{equation*}
			a_{ij}=\begin{cases}
				0&(i<j),\\
				-1&(i=j),\\
				2^{j-i}&(i>j).
			\end{cases}
		\end{equation*}
		Prove that
		\begin{equation*}
			\sum_i \sum_j a_{ij}=-2,\qquad \sum_j\sum_i a_{ij}=0.
		\end{equation*}
	\end{exe}
	\begin{proof}
		First, fix $i$, and define $\sum_j a_{ij}=b_i$.\\
		By definition, $a_{ij}=0$ if $i<j$, we are enough to calculate the value for $i\geq j$.\\
		As $a_{ij}=2^{j-i}$ if $i>j$, $$b_i=\sum_{j=1} ^ {i-1} 2^{j-i}-1=\dfrac{2^{1-i}(2^{i-1}-1)}{2-1}-1=-2^{1-i}.$$
		Thus $$\displaystyle\sum_i \sum_j a_{ij}=\sum_i b_i=\sum_i -2^{1-i}=-2.$$
		In the same way, fix $j$, and define $\sum_i a_{ij}=c_j$.\\
		Then, $$c_j=\sum_{i=j+1}^{\infty} 2^{j-i} -1=\dfrac{1/2}{1-1/2}-1=1-1=0.$$
		Thus, $$\displaystyle\sum_j \sum_i a_{ij}=\sum_j c_j = 0.$$
	\end{proof}
	\newpage
	\begin{exe}
		Prove that
		\begin{equation*}
			\sum_i \sum_j a_{ij}=\sum_j\sum_i a_{ij}
		\end{equation*}
		if $a_{ij}\geq0$ for all $i$ and $j$ (the case $+\infty=+\infty$ may occur).
	\end{exe}
	\begin{proof}
		First,  x
	\end{proof}
	\newpage
	\setcounter{exe}{10}
	\begin{exe}
		Suppose $f\in\mathscr{R}$ on $[0,\ A]$ for all $A<\infty$, and $f(x)\rightarrow 1$ as $x\rightarrow +\infty$. Prove that $$\displaystyle\lim_{t\rightarrow 0}\ t\int_{0}^{\infty} e^{-tx}f(x)\,dx=1\qquad (t>0).$$
	\end{exe}
	\begin{proof}
		Note that $f\in\mathscr{R}$ on $[0,\ A]$ for all $A<\infty$, so we don't guarantee about existence of $\displaystyle\int_A^\infty f(x)dx$.\\
		Also we can guarantee about $\displaystyle\int_0 ^A |f(x)| dx$, let assume this value $k$.\\
		Since $h(x)=e^{-tx}$ be strictly increasing function for every $t>0$,$$\left\vert\int_0^A e^{-tx}f(x) dx\right\vert \leq \int_0^A e^{-tx}|f(x)|dx\leq e^{-tA}\int_0^A |f(x)|dx=ke^{-tA}$$by \textbf{Theorem 6.12(b), 6.13}.\\
		By above, we can easily know that $$\lim_{t\rightarrow 0} t\int_0^A e^{-tx}f(x)dx=0,$$ so if we show that $\displaystyle\lim_{t\rightarrow 0}\int_A^\infty e^{-tx}f(x)dx$ exists, which value is $1$, proof is completed.\\
		Let $\varepsilon>0$ given.\\
		Since $f(x)\rightarrow 1$ as $x\rightarrow\infty$, there exists $t$ s.t. $|f(x)-1|\leq \varepsilon$ for all $x\geq t$.\\
		Letting $A=t$, then $1-\varepsilon\leq f(x)\leq 1+\varepsilon$ for all $x>A$, and $f\in\mathscr{R}$ on $[0,\,A]$.\\
		Also $h(x)=e^{-tx}$ goes to $0$ as $x\rightarrow \infty$ for $t>0$, $\displaystyle \int_a^\infty Kte^{-tx}dx$ well defined, and value is $Ke^{-ta}$.
		Thus $1-\varepsilon\leq f(x)\leq 1+\varepsilon$ for every $x>A$, \begin{equation}
			(1-\varepsilon)e^{-tA}\leq\underline{\int_A^\infty} te^{-tx}f(x)dx\leq\overline{\int_A^\infty} te^{-tx}f(x)dx\leq (1+\varepsilon)e^{-tA}
		\end{equation}
		Letting $t\rightarrow 0$, $e^{-tA}$ goes to $1$, so left and right side on equation (1) goes to $1-\varepsilon$ and $1+\varepsilon$, respectively.\\
		This shows that $\displaystyle\int_a^b te^{-tx}f(x)dx=1$, and completes the proof.
	\end{proof}
	\newpage
	\setcounter{exe}{12}
	\begin{exe}
		Put $f(x)=x$ if $0\leq x < 2\pi$, and apply Parseval's theorem to conclude that $$\displaystyle\sum_{n=1}^{\infty} \dfrac{1}{n^2}=\dfrac{\pi^2}{6}.$$
	\end{exe}
	\begin{proof}
	\end{proof}
	\newpage
	\begin{exe}
		If $f(x)=(\pi-|x|)^2$ on $[-\pi,~\pi]$, prove that $$f(x)=\dfrac{\pi^2}{3}+\displaystyle\sum_{n=1}^\infty \dfrac{4}{n^2}\,\cos nx$$ and deduce that $$\displaystyle\sum_{n=1}^\infty \dfrac{1}{n^2}=\dfrac{\pi^2}{6},\qquad \sum_{n=1}^{\infty} \dfrac{1}{n^4}=\dfrac{\pi^4}{90}.$$
	\end{exe}
	\begin{proof}
	\end{proof}
	\newpage
	\setcounter{exe}{21}
	\begin{exe}
		If $\alpha$ is real and $-1<x<1$, prove Newton's binomial theorem $$(1+x)^\alpha=1+\displaystyle\sum_{n=1}^\infty \dfrac{\alpha(\alpha-1)\cdots(\alpha-n+1)}{n!}x^n.$$
		$Hint:$ Denote the right side by $f(x)$. Prove that the series converges.\\
		Prove that $$(1+x)f\,'(x)=\alpha f(x)$$ and solve this differential equation.\\
		Show also that $$(1-x)^{-\alpha}=\displaystyle\sum_{n=0}^\infty \dfrac{\Gamma(n+\alpha)}{n!\,\Gamma(\alpha)} x^n$$ if $-1<x<1$ and $\alpha>0$.
	\end{exe}
	\begin{proof}
	\end{proof}
\end{document}